
%----------------------------------------------------------------------------------------
%	Package et configuration
%----------------------------------------------------------------------------------------

\documentclass[11pt,a4paper,sans]{moderncv} % Font sizes: 10, 11, or 12; paper sizes: a4paper, letterpaper, a5paper, legalpaper, executivepaper or landscape; font families: sans or roman

\moderncvstyle{classic} % CV theme - options include: 'casual' (default), 'classic', 'oldstyle' and 'banking'
\moderncvcolor{blue} % CV color - options include: 'blue' (default), 'orange', 'green', 'red', 'purple', 'grey' and 'black'

\usepackage{soul}
\usepackage[scale=0.75]{geometry} % Reduce document margins
\usepackage{lastpage}
\usepackage{fancyhdr}
\fancyfoot[C]{Page \thepage\ de \pageref{LastPage}}
\pagestyle{fancy}

%----------------------------------------------------------------------------------------
%	Nom et contact
%----------------------------------------------------------------------------------------

\firstname{Adrien}
\familyname{Cloutier}

\title{}
\address{401, rue de la Tourelle, Québec, Qc}{}
\mobile{Téléphone: 418-590-2605}
\email{ADCLO2@ulaval.ca}

%----------------------------------------------------------------------------------------

\begin{document}

\makecvtitle

%----------------------------------------------------------------------------------------
%	Formation
%----------------------------------------------------------------------------------------

\section{\textbf{Formations}}

\cvitem{Depuis 2023}{\textbf{Université Laval}, Québec}
\cvitem{}{\textbf{Candidat au doctorat en science politique}, Directeur: \textit{Yannick Dufresne, Ph.D.}}
\cvitem{}{\textit{Intérêts}: Communication politique; Opinion publique; Journalisme; Analyse textuelle.}
\vspace{1em}

\cvitem{2018-2020}{\textbf{Université Laval}, Québec}
\cvitem{}{\textbf{Maitrise en science politique}, Directeur: \textit{Yannick Dufresne, Ph.D.}}
\vspace{1em}

\cvitem{2019}{\textbf{University of Michigan}, Michigan}
\cvitem{}{\textbf{Summer Program in Quantitative Methods of Social Research}}
\cvitem{}{Inter-university Consortium for Political and Social Research (ICPSR)}
\vspace{1em}

\cvitem{2015 -- 2018}{\textbf{Université Laval}, Québec}
\cvitem{}{\textbf{Baccalauréat en science politique}}
\vspace{1em}

\cvitem{2012 -- 2015}{\textbf{Cégep de Jonquière}, Jonquière}
\cvitem{}{\textbf{Technique en Art et technologie des médias (ATM)}}
\cvitem{}{\textit{Profil}: Journalisme}
\vspace{1em}

%----------------------------------------------------------------------------------------
%	Publications
%----------------------------------------------------------------------------------------

\section{\textbf{Publications}}

\cvitem{2023}{Ouimet, M., Beaumier, M., \textbf{Adrien Cloutier}, Côté, A., Montigny, E., Gélineau, F., Jacob, S., et Ratté, S.. \textit{Use of research evidence in legislatures: a systematic review}. Evidence and Policy, 1-18.}
\vspace{1em}

\cvitem{2022}{\textbf{Adrien Cloutier}, Tremblay‐Antoine, C., Dufresne, Y., et Fréchet, N.. \textit{Highs and downs: A scoping review of public opinion about cannabis, alcohol and tobacco in Canada}. Drug and alcohol review, 41(2), 396-405.}
\vspace{1em}

\cvitem{2022}{Tremblay-Antoine, C., Dufresne, Y., Fortier-Chouinard, \textbf{Adrien Cloutier}, et Côté, A.. \textit{L’évolution de la saillance des enjeux et le bilan du gouvernement de la CAQ: un éclairage nouveau}. Bilan du gouvernement de la CAQ: Entre nationalisme et pandémie, 67.}
\vspace{1em}

\cvitem{2022}{Tomkinson, S. et \textbf{Adrien Cloutier}. \textit{Représentations médiatiques des jeunes demandeurs d’asile irréguliers au Québec et au Canada}. Hommes et Migrations, 99-106.}
\vspace{1em}

\cvitem{2021}{Bibeau, A., \textbf{Adrien Cloutier}, Fortier-Chouinard, A., Fréchet, N., Tremblay-Antoine, C., et Dufresne, Y.. \textit{Positive communication in a catastrophic crisis: The mixed effects of COVID-19 on the tone of Canadian governments’ media coverage}. International journal of media and cultural politics, 17(1), 69-79.}
\vspace{1em}

\cvitem{2019}{\textbf{Adrien Cloutier} et Montigny, E.. \textit{As the push for provincial autonomy spreads, where will it lead?} Policy Options.}
\vspace{1em}

%----------------------------------------------------------------------------------------
%	 Conférences
%----------------------------------------------------------------------------------------

\section{\textbf{Conférences}}

\cvitem{2023}{Congrès de la Southern Political Science Association (SPSA). \textit{Good and Bad Migrants? Asymmetric Empathy in the News Coverage of the Syrian and Ukrainian Migrant Crises.} St. Pete Beach, Florida.}
\vspace{1em}

\cvitem{2020}{Congrès de la Southern Political Science Association (SPSA). \textit{Cannabis, Alcohol and Tobacco: A Scoping Review of the “High” and “Down” of Canadian Public Opinion.} San Juan, Puerto Rico.}
\vspace{1em}

\cvitem{2019}{Congrès de l'Association canadienne de science politique (ACSP). \textit{Managing the “Crisis” of Irregular Border Crossings: Media Narratives and Policy Responses in Canada}. University of British Columbia (UBC), Vancouver.}
\vspace{1em}

%----------------------------------------------------------------------------------------
%	Contrats
%----------------------------------------------------------------------------------------

%----------------------------------------------------------------------------------------
%	Autres expériences
%----------------------------------------------------------------------------------------

\section{\textbf{Expériences professionnelles}}

\cvitem{2020-23}{\textbf{Auxiliaire d'enseignement}}
\cvitem{}{\textit{Analyse quantitative (POL-7004).}}
\vspace{1em}

\cvitem{2019}{\textbf{Auxiliaire d'enseignement}}
\cvitem{}{\textit{Persuasion sociale, influence et opinion publique (POL-2425).}}
\vspace{1em}

\cvitem{2019}{\textbf{Concours national d’études de cas en administration publique}}
\cvitem{}{\textit{Association canadienne des programmes en administration publiques (ACPAP).}}
\vspace{1em}

%----------------------------------------------------------------------------------------
%	Groupe de recherche
%----------------------------------------------------------------------------------------

\section{\textbf{Groupes de recherche}}

\cventry{Depuis 2019}{Membre de l'Observatoire international sur les impacts sociétaux de l'IA et du numérique (OBVIA)}{Université Laval}{Québec}{}
{Chercheur étudiant.}
\vspace{1em}

\cventry{Depuis 2019}{Membre de la Chaire de leadership en enseignement des sciences sociales numérique (CLESSN)}{Université Laval}{Québec}{}
{Chercheur étudiant.}
\vspace{1em}

\cventry{Depuis 2019}{Membre du Centre d'analyse des politiques publiques (CAPP)}{Université Laval}{Québec}{}
{Chercheur étudiant.}
\vspace{1em}

\cventry{Depuis 2018}{Membre du Groupe de recherche en communication politique (GRCP)}{Université Laval}{Québec}{}
{Chercheur étudiant.}
\vspace{1em}

\cventry{Depuis 2018}{Membre du Centre pour l’étude de la citoyenneté démocratique (CÉCD-CSDC)}{Université Laval}{Québec}{}
{Chercheur étudiant.}
\vspace{1em}

%----------------------------------------------------------------------------------------
%	Bourses et prix
%----------------------------------------------------------------------------------------

\section{\textbf{Bourses et reconnaissances}}

\cventry{2020-23}{Bourse au doctorat en recherche}{Fonds de recherche Société et culture (FRQSC)}{Université Laval}{}
{}

\cventry{2020}{Bourse de développement méthodologique}{Chaire de leadership en enseignement des sciences sociales numériques}{Université Laval}{}
{}

\cventry{2019}{Bourse de formation méthodologique}{Centre pour l’étude de la citoyenneté démocratique}{Université McGill}{}
{}

\cventry{2019}{Bourse de formation méthodologique}{Groupe de recherche en communication politique}{Université Laval}{}
{}

\cventry{2019}{Bourse de formation en méthodes qualitatives ou quantitatives}{Département de science politique}{Université Laval}{}
{}

\cventry{2019}{Bourse de formation en R}{Département de science politique}{Université Laval}{}
{}

\cventry{2018}{Bourse de mérite}{Chaire de leadership en enseignement des sciences sociales numériques}{Université Laval}{}
{}


\vspace{1em}

%----------------------------------------------------------------------------------------
%	Skillsssss
%----------------------------------------------------------------------------------------

\section{\textbf{Compétences spécifiques}}

\cvitem{Programmes}{Langages LateX et Markdown, R, HTML, Git/Github, Notion, Vim, Suite Adobe, Série Microsoft Office.}
\vspace{1em}

\cvitem{Spécialisation}{Analyse textuelle automatisée (Bag of Words, Topic Modeling).}
\vspace{1em}

\cvitem{Langues}{Français et anglais.}
\vspace{1em}

%----------------------------------------------------------------------------------------
%	Références
%----------------------------------------------------------------------------------------

 \section{\textbf{Références}}

 Fournies sur demande.

%----------------------------------------------------------------------------------------

\end{document}
