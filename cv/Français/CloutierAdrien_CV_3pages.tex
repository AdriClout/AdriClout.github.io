
%----------------------------------------------------------------------------------------
%	Package et configuration
%----------------------------------------------------------------------------------------

\documentclass[11pt,a4paper,sans]{moderncv} % Font sizes: 10, 11, or 12; paper sizes: a4paper, letterpaper, a5paper, legalpaper, executivepaper or landscape; font families: sans or roman

\moderncvstyle{classic} % CV theme - options include: 'casual' (default), 'classic', 'oldstyle' and 'banking'
\moderncvcolor{blue} % CV color - options include: 'blue' (default), 'orange', 'green', 'red', 'purple', 'grey' and 'black'

\usepackage{soul}
\usepackage{lmodern}
\usepackage[scale=0.75]{geometry} % Reduce document margins
\usepackage{lastpage}
\usepackage{fancyhdr}
\usepackage{hyperref}
\fancyfoot[C]{Page \thepage\ de \pageref{LastPage}}
\pagestyle{fancy}

%----------------------------------------------------------------------------------------
%	Nom et contact
%----------------------------------------------------------------------------------------

\firstname{Adrien}
\familyname{Cloutier}

\title{Doctorant en science politique | Université Laval}
\address{923 rue des pommes}{}
\mobile{Téléphone: 418-590-2605}
\email{ADCLO2@ulaval.ca}
\social[googlescholar]{RXAdvoMAAAAJ}
\homepage{adriencloutier.com}

%----------------------------------------------------------------------------------------

\begin{document}

\makecvtitle

%----------------------------------------------------------------------------------------
%	Formation
%----------------------------------------------------------------------------------------

\section{\textbf{Formations}}

\cvitem{Depuis 2023}{\textbf{Université Laval}, Québec}
\cvitem{}{\textbf{Candidat au doctorat en science politique}, Directeur: \textit{Yannick Dufresne, Ph.D.}}
\cvitem{}{\textit{Thèse}: «Breaking News: Radar+ for Measuring Media Salience in the Digital Age»}
\cvitem{}{\textit{Défense prévue}: Automne 2026}
\cvitem{}{\textit{Financement}: Bourse doctorale FRQ (Fonds de recherche Société et culture)}
\cvitem{}{\textit{Champs de spécialisation}: Communication politique; Agenda-setting; Saillance médiatique; Polarisation; Méthodes computationnelles; Science ouverte.}
\vspace{1em}

\cvitem{2018-2020}{\textbf{Université Laval}, Québec}
\cvitem{}{\textbf{Maitrise en science politique}, Directeur: \textit{Yannick Dufresne, Ph.D.}}
\vspace{1em}

\cvitem{2019}{\textbf{University of Michigan}, Michigan}
\cvitem{}{\textbf{Summer Program in Quantitative Methods of Social Research}}
\cvitem{}{Inter-university Consortium for Political and Social Research (ICPSR)}
\vspace{1em}

\cvitem{2015 -- 2018}{\textbf{Université Laval}, Québec}
\cvitem{}{\textbf{Baccalauréat en science politique}}
\vspace{1em}

\cvitem{2012 -- 2015}{\textbf{Cégep de Jonquière}, Jonquière}
\cvitem{}{\textbf{Technique en Art et technologie des médias (ATM)}}
\cvitem{}{\textit{Profil}: Journalisme}
\vspace{1em}

%----------------------------------------------------------------------------------------
%	Recherche en cours et projets
%----------------------------------------------------------------------------------------

\section{\textbf{Recherche en cours}}

\cventry{2019--2026}{Projet doctoral: Radar+}{Université Laval}{}{}
{Infrastructure computationnelle développée en langage R pour l'archivage automatisé des pages d'accueil de ~20 médias québécois, canadiens et américains (collecte toutes les 10 minutes depuis 2019). Développement d'indices de saillance absolue et relative (0--100) via modèle de langage local (LLM). Identification d'objets médiatiques (enjeux, acteurs, institutions, lieux). \\
\textit{Lien}: \url{https://www.clessn.com/radar/index.html}}
\vspace{1em}

\cventry{2027--}{Projet postdoctoral: Fox vs CNN}{UCLA (University of California, Los Angeles)}{}{}
{«Fox vs CNN: A Longitudinal Analysis of Media Salience and Polarization in the United States». Supervision: Prof. Stuart Soroka. Analyse longitudinale (2019--2027) du paysage médiatique américain. Développement d'un indice de divergence/concordance médiatique. Production d'articles scientifiques, visualisations publiques et tableau de bord ouvert.}
\vspace{1em}

%----------------------------------------------------------------------------------------
%	Publications
%----------------------------------------------------------------------------------------

\section{\textbf{Publications}}

\cvitem{}{Profil Google Scholar: \url{https://scholar.google.com/citations?user=RXAdvoMAAAAJ&hl=fr}}
\vspace{0.5em}

\cvitem{2023}{Ouimet, M., Beaumier, M., \textbf{Adrien Cloutier}, Côté, A., Montigny, E., Gélineau, F., Jacob, S., et Ratté, S.. \textit{Use of research evidence in legislatures: a systematic review}. Evidence and Policy, 1-18.}
\vspace{1em}

\cvitem{2022}{\textbf{Adrien Cloutier}, Tremblay‐Antoine, C., Dufresne, Y., et Fréchet, N.. \textit{Highs and downs: A scoping review of public opinion about cannabis, alcohol and tobacco in Canada}. Drug and alcohol review, 41(2), 396-405.}
\vspace{1em}

\cvitem{2022}{Tremblay-Antoine, C., Dufresne, Y., Fortier-Chouinard, \textbf{Adrien Cloutier}, et Côté, A.. \textit{L'évolution de la saillance des enjeux et le bilan du gouvernement de la CAQ: un éclairage nouveau}. Bilan du gouvernement de la CAQ: Entre nationalisme et pandémie, 67.}
\vspace{1em}

\cvitem{2022}{Tomkinson, S. et \textbf{Adrien Cloutier}. \textit{Représentations médiatiques des jeunes demandeurs d'asile irréguliers au Québec et au Canada}. Hommes et Migrations, 99-106.}
\vspace{1em}

\cvitem{2021}{Bibeau, A., \textbf{Adrien Cloutier}, Fortier-Chouinard, A., Fréchet, N., Tremblay-Antoine, C., et Dufresne, Y.. \textit{Positive communication in a catastrophic crisis: The mixed effects of COVID-19 on the tone of Canadian governments' media coverage}. International journal of media and cultural politics, 17(1), 69-79.}
\vspace{1em}

\cvitem{2019}{\textbf{Adrien Cloutier} et Montigny, E.. \textit{As the push for provincial autonomy spreads, where will it lead?} Policy Options.}
\vspace{1em}

%----------------------------------------------------------------------------------------
%	 Conférences
%----------------------------------------------------------------------------------------

\section{\textbf{Conférences}}

\cvitem{2023}{Congrès de la Southern Political Science Association (SPSA). \textit{Good and Bad Migrants? Asymmetric Empathy in the News Coverage of the Syrian and Ukrainian Migrant Crises.} St. Pete Beach, Florida.}
\vspace{1em}

\cvitem{2020}{Congrès de la Southern Political Science Association (SPSA). \textit{Cannabis, Alcohol and Tobacco: A Scoping Review of the "High" and "Down" of Canadian Public Opinion.} San Juan, Puerto Rico.}
\vspace{1em}

\cvitem{2019}{Congrès de l'Association canadienne de science politique (ACSP). \textit{Managing the "Crisis" of Irregular Border Crossings: Media Narratives and Policy Responses in Canada}. University of British Columbia (UBC), Vancouver.}
\vspace{1em}

%----------------------------------------------------------------------------------------
%	Expériences professionnelles
%----------------------------------------------------------------------------------------

\section{\textbf{Expériences professionnelles}}

\cventry{Depuis 2024}{Chargé de cours}{Université Laval}{Québec}{}
{Création et enseignement du cours «Outils numériques en sciences sociales» (POL-6078).}
\vspace{1em}

\cventry{2022--2024}{Coordonnateur scientifique}{Réseau francophone international en conseil scientifique (RFICS)}{}{}
{Coordination scientifique d'un réseau interdisciplinaire et international reliant chercheurs, parlementaires et décideurs publics. Gestion d'équipes internationales. Gouvernance (comités, réunions, processus décisionnels). Production de documents stratégiques. Organisation d'un atelier international à Dakar (11--12 juillet 2024). Rédaction de termes de référence pour formations en conseil scientifique destinées à des parlementaires. Gestion de partenariats (ex. CODE-Africa). \\
\textit{Lien}: \url{https://rfics.org}}
\vspace{1em}

\cventry{2022 et 2024}{Chercheur Mitacs}{Musée de la Civilisation}{Québec}{}
{Deux projets Mitacs (15 000\$ chacun). Nettoyage de données, analyses croisées, développement de scripts reproductibles en R, création de tableaux de bord et rapports d'aide à la décision.}
\vspace{1em}

\cventry{Depuis 2019}{Co-créateur et co-organisateur}{École interdisciplinaire outils \& méthodes (EIOM)}{}{}
{École d'été annuelle dédiée aux meilleures pratiques en recherche, réunissant étudiants, chercheurs et professionnels. \\
\textit{Lien}: \url{https://eiom.ca}}
\vspace{1em}

\cventry{2020-2023}{Auxiliaire d'enseignement}{Université Laval}{Québec}{}
{Cours: Analyse quantitative (POL-7004) et Méthodologie quantitative (baccalauréat).}
\vspace{1em}

\cventry{2019}{Auxiliaire d'enseignement}{Université Laval}{Québec}{}
{Cours: Persuasion sociale, influence et opinion publique (POL-2425).}
\vspace{1em}

\cventry{2019}{Concours national d'études de cas en administration publique}{Association canadienne des programmes en administration publiques (ACPAP)}{}{}
{}
\vspace{1em}

%----------------------------------------------------------------------------------------
%	Groupe de recherche
%----------------------------------------------------------------------------------------

\section{\textbf{Groupes de recherche}}

\cventry{Depuis 2019}{Membre de l'Observatoire international sur les impacts sociétaux de l'IA et du numérique (OBVIA)}{Université Laval}{Québec}{}
{Chercheur étudiant.}
\vspace{1em}

\cventry{Depuis 2019}{Membre de la Chaire de leadership en enseignement des sciences sociales numérique (CLESSN)}{Université Laval}{Québec}{}
{Chercheur étudiant.}
\vspace{1em}

\cventry{Depuis 2019}{Membre du Centre d'analyse des politiques publiques (CAPP)}{Université Laval}{Québec}{}
{Chercheur étudiant.}
\vspace{1em}

\cventry{Depuis 2018}{Membre du Groupe de recherche en communication politique (GRCP)}{Université Laval}{Québec}{}
{Chercheur étudiant.}
\vspace{1em}

\cventry{Depuis 2018}{Membre du Centre pour l'étude de la citoyenneté démocratique (CÉCD-CSDC)}{Université Laval}{Québec}{}
{Chercheur étudiant.}
\vspace{1em}

%----------------------------------------------------------------------------------------
%	Bourses et prix
%----------------------------------------------------------------------------------------

\section{\textbf{Bourses et reconnaissances}}

\cventry{2020-2023}{Bourse au doctorat en recherche (B2X)}{Fonds de recherche Société et culture (FRQSC)}{Université Laval}{}
{Financement pour le projet doctoral Radar+.}

\cventry{2024}{Bourse d'excellence du RFICS}{Réseau francophone international en conseil scientifique}{2 000\$}{}
{}

\cventry{2020}{Bourse de développement méthodologique}{Chaire de leadership en enseignement des sciences sociales numériques}{Université Laval}{}
{}

\cventry{2019}{Bourse de formation méthodologique}{Centre pour l'étude de la citoyenneté démocratique}{Université McGill}{}
{}

\cventry{2019}{Bourse de formation méthodologique}{Groupe de recherche en communication politique}{Université Laval}{}
{}

\cventry{2019}{Bourse de formation en méthodes qualitatives ou quantitatives}{Département de science politique}{Université Laval}{}
{}

\cventry{2019}{Bourse de formation en R}{Département de science politique}{Université Laval}{}
{}

\cventry{2018}{Bourse de mérite}{Chaire de leadership en enseignement des sciences sociales numériques}{Université Laval}{}
{}


\vspace{1em}

%----------------------------------------------------------------------------------------
%	Compétences spécifiques
%----------------------------------------------------------------------------------------

\section{\textbf{Compétences spécifiques}}

\cvitem{Langages de programmation}{R (avancé), HTML, CSS, LaTeX, Markdown}
\vspace{1em}

\cvitem{Outils et logiciels}{Git/Github, Notion (créateur de templates publics), Vim, Suite Adobe, Microsoft Office}
\vspace{1em}

\cvitem{Méthodologies}{Analyse textuelle automatisée (Bag of Words, Topic Modeling, LLM), Méthodes computationnelles, Science ouverte}
\vspace{1em}

\cvitem{Gestion de projet}{Scrum Master certifié, Gestion agile, Templates Notion (\url{https://www.notion.com/@adri01})}
\vspace{1em}

\cvitem{Langues}{Français (langue maternelle) et Anglais (courant)}
\vspace{1em}

%----------------------------------------------------------------------------------------
%	Références
%----------------------------------------------------------------------------------------

 \section{\textbf{Références}}

 Fournies sur demande.

%----------------------------------------------------------------------------------------

\end{document}
