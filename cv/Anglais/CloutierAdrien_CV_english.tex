
%----------------------------------------------------------------------------------------
%	Package et configuration
%----------------------------------------------------------------------------------------

\documentclass[11pt,a4paper,sans]{moderncv} % Font sizes: 10, 11, or 12; paper sizes: a4paper, letterpaper, a5paper, legalpaper, executivepaper or landscape; font families: sans or roman

\moderncvstyle{classic} % CV theme - options include: 'casual' (default), 'classic', 'oldstyle' and 'banking'
\moderncvcolor{blue} % CV color - options include: 'blue' (default), 'orange', 'green', 'red', 'purple', 'grey' and 'black'

\usepackage{soul}
\usepackage{lmodern}
\usepackage[scale=0.75]{geometry} % Reduce document margins
\usepackage{lastpage}
\usepackage{fancyhdr}
\usepackage{hyperref}
\fancyfoot[C]{Page \thepage\ of \pageref{LastPage}}
\pagestyle{fancy}

%----------------------------------------------------------------------------------------
%	Name and contact
%----------------------------------------------------------------------------------------

\firstname{Adrien}
\familyname{Cloutier}

\title{PhD Candidate in Political Science | Université Laval}
\address{923 rue des pommes}{}
\mobile{Phone: 418-590-2605}
\email{ADCLO2@ulaval.ca}
\social[googlescholar]{RXAdvoMAAAAJ}
\homepage{adriencloutier.com}

%----------------------------------------------------------------------------------------

\begin{document}

\makecvtitle

%----------------------------------------------------------------------------------------
%	Education
%----------------------------------------------------------------------------------------

\section{\textbf{Education}}

\cvitem{Since 2023}{\textbf{Université Laval}, Québec}
\cvitem{}{\textbf{PhD Candidate in Political Science}, Supervisor: \textit{Yannick Dufresne, Ph.D.}}
\cvitem{}{\textit{Dissertation}: "Breaking News: Radar+ for Measuring Media Salience in the Digital Age"}
\cvitem{}{\textit{Defense}: Fall 2026}
\cvitem{}{\textit{Funding}: FRQ Doctoral Scholarship (Fonds de recherche Société et culture)}
\cvitem{}{\textit{Fields of specialization}: Political communication; Agenda-setting; Media salience; Polarization; Computational methods; Open science.}
\vspace{1em}

\cvitem{2018-2020}{\textbf{Université Laval}, Québec}
\cvitem{}{\textbf{Master's in Political Science}, Supervisor: \textit{Yannick Dufresne, Ph.D.}}
\vspace{1em}

\cvitem{2019}{\textbf{University of Michigan}, Michigan}
\cvitem{}{\textbf{Summer Program in Quantitative Methods of Social Research}}
\cvitem{}{Inter-university Consortium for Political and Social Research (ICPSR)}
\vspace{1em}

\cvitem{2015 -- 2018}{\textbf{Université Laval}, Québec}
\cvitem{}{\textbf{Bachelor's in Political Science}}
\vspace{1em}

\cvitem{2012 -- 2015}{\textbf{Cégep de Jonquière}, Jonquière}
\cvitem{}{\textbf{Technique in Art and Media Technology (ATM)}}
\cvitem{}{\textit{Profile}: Journalism}
\vspace{1em}

%----------------------------------------------------------------------------------------
%	Current Research and Projects
%----------------------------------------------------------------------------------------

\section{\textbf{Current Research}}

\cventry{2019--2026}{Doctoral Project: Radar+}{Université Laval}{}{}
{Computational infrastructure developed in R language for automated archiving of homepages from ~20 Québécois, Canadian, and American media outlets (data collection every 10 minutes since 2019). Development of absolute and relative salience indices (0--100) using local language models (LLM). Identification of media objects (issues, actors, institutions, locations). \\
\textit{Link}: \url{https://www.clessn.com/radar/index.html}}
\vspace{1em}

\cventry{2027--}{Postdoctoral Project: Fox vs CNN}{UCLA (University of California, Los Angeles)}{}{}
{"Fox vs CNN: A Longitudinal Analysis of Media Salience and Polarization in the United States". Supervision: Prof. Stuart Soroka. Longitudinal analysis (2019--2027) of the American media landscape. Development of a media divergence/convergence index. Production of scientific articles, public visualizations, and open dashboard.}
\vspace{1em}

%----------------------------------------------------------------------------------------
%	Publications
%----------------------------------------------------------------------------------------

\section{\textbf{Publications}}

\cvitem{}{Google Scholar Profile: \url{https://scholar.google.com/citations?user=RXAdvoMAAAAJ&hl=fr}}
\vspace{0.5em}

\cvitem{2023}{Ouimet, M., Beaumier, M., \textbf{Adrien Cloutier}, Côté, A., Montigny, E., Gélineau, F., Jacob, S., et Ratté, S.. \textit{Use of research evidence in legislatures: a systematic review}. Evidence and Policy, 1-18.}
\vspace{1em}

\cvitem{2022}{\textbf{Adrien Cloutier}, Tremblay‐Antoine, C., Dufresne, Y., et Fréchet, N.. \textit{Highs and downs: A scoping review of public opinion about cannabis, alcohol and tobacco in Canada}. Drug and alcohol review, 41(2), 396-405.}
\vspace{1em}

\cvitem{2022}{Tremblay-Antoine, C., Dufresne, Y., Fortier-Chouinard, \textbf{Adrien Cloutier}, et Côté, A.. \textit{L'évolution de la saillance des enjeux et le bilan du gouvernement de la CAQ: un éclairage nouveau}. Bilan du gouvernement de la CAQ: Entre nationalisme et pandémie, 67.}
\vspace{1em}

\cvitem{2022}{Tomkinson, S. et \textbf{Adrien Cloutier}. \textit{Représentations médiatiques des jeunes demandeurs d'asile irréguliers au Québec et au Canada}. Hommes et Migrations, 99-106.}
\vspace{1em}

\cvitem{2021}{Bibeau, A., \textbf{Adrien Cloutier}, Fortier-Chouinard, A., Fréchet, N., Tremblay-Antoine, C., et Dufresne, Y.. \textit{Positive communication in a catastrophic crisis: The mixed effects of COVID-19 on the tone of Canadian governments' media coverage}. International journal of media and cultural politics, 17(1), 69-79.}
\vspace{1em}

\cvitem{2019}{\textbf{Adrien Cloutier} et Montigny, E.. \textit{As the push for provincial autonomy spreads, where will it lead?} Policy Options.}
\vspace{1em}

%----------------------------------------------------------------------------------------
%	Conference Presentations
%----------------------------------------------------------------------------------------

\section{\textbf{Conference Presentations}}

\cvitem{2023}{Southern Political Science Association (SPSA) Conference. \textit{Good and Bad Migrants? Asymmetric Empathy in the News Coverage of the Syrian and Ukrainian Migrant Crises.} St. Pete Beach, Florida.}
\vspace{1em}

\cvitem{2020}{Southern Political Science Association (SPSA) Conference. \textit{Cannabis, Alcohol and Tobacco: A Scoping Review of the "High" and "Down" of Canadian Public Opinion.} San Juan, Puerto Rico.}
\vspace{1em}

\cvitem{2019}{Canadian Political Science Association (CPSA) Conference. \textit{Managing the "Crisis" of Irregular Border Crossings: Media Narratives and Policy Responses in Canada}. University of British Columbia (UBC), Vancouver.}
\vspace{1em}

%----------------------------------------------------------------------------------------
%	Professional Experience
%----------------------------------------------------------------------------------------

\section{\textbf{Professional Experience}}

\cventry{Since 2024}{Lecturer (Chargé de cours)}{Université Laval}{Québec}{}
{Creation and teaching of the course "Digital Tools in Social Sciences" (POL-6078).}
\vspace{1em}

\cventry{2022--2024}{Scientific Coordinator}{Réseau francophone international en conseil scientifique (RFICS)}{}{}
{Scientific coordination of an interdisciplinary and international network connecting researchers, parliamentarians, and public decision-makers. Management of international teams. Governance (committees, meetings, decision-making processes). Production of strategic documents. Organization of an international workshop in Dakar (July 11--12, 2024). Drafting of terms of reference for science advisory training for parliamentarians. Partnership management (e.g., CODE-Africa). \\
\textit{Link}: \url{https://rfics.org}}
\vspace{1em}

\cventry{2022 and 2024}{Mitacs Researcher}{Musée de la Civilisation}{Québec}{}
{Two Mitacs projects (\$15,000 each). Data cleaning, cross-sectional analyses, development of reproducible R scripts, creation of dashboards and decision-support reports.}
\vspace{1em}

\cventry{Since 2019}{Co-creator and Co-organizer}{École interdisciplinaire outils \& méthodes (EIOM)}{}{}
{Annual summer school dedicated to research best practices, bringing together students, researchers, and professionals. \\
\textit{Link}: \url{https://eiom.ca}}
\vspace{1em}

\cventry{2020-2023}{Teaching Assistant}{Université Laval}{Québec}{}
{Courses: Quantitative Analysis (POL-7004) and Quantitative Methodology (undergraduate).}
\vspace{1em}

\cventry{2019}{Teaching Assistant}{Université Laval}{Québec}{}
{Course: Social Persuasion, Influence and Public Opinion (POL-2425).}
\vspace{1em}

\cventry{2019}{National Case Study Competition in Public Administration}{Canadian Association of Programs in Public Administration (CAPPA)}{}{}
{}
\vspace{1em}

%----------------------------------------------------------------------------------------
%	Research Groups
%----------------------------------------------------------------------------------------

\section{\textbf{Research Groups}}

\cventry{Since 2019}{Member of the Observatoire international sur les impacts sociétaux de l'IA et du numérique (OBVIA)}{Université Laval}{Québec}{}
{Student Researcher.}
\vspace{1em}

\cventry{Since 2019}{Member of the Chaire de leadership en enseignement des sciences sociales numérique (CLESSN)}{Université Laval}{Québec}{}
{Student Researcher.}
\vspace{1em}

\cventry{Since 2019}{Member of the Centre d'analyse des politiques publiques (CAPP)}{Université Laval}{Québec}{}
{Student Researcher.}
\vspace{1em}

\cventry{Since 2018}{Member of the Groupe de recherche en communication politique (GRCP)}{Université Laval}{Québec}{}
{Student Researcher.}
\vspace{1em}

\cventry{Since 2018}{Member of the Centre pour l'étude de la citoyenneté démocratique (CÉCD-CSDC)}{Université Laval}{Québec}{}
{Student Researcher.}
\vspace{1em}

%----------------------------------------------------------------------------------------
%	Scholarships and Awards
%----------------------------------------------------------------------------------------

\section{\textbf{Scholarships and Awards}}

\cventry{2020-2023}{Doctoral Research Scholarship (B2X)}{Fonds de recherche Société et culture (FRQSC)}{Université Laval}{}
{Funding for the Radar+ doctoral project.}

\cventry{2024}{RFICS Excellence Scholarship}{Réseau francophone international en conseil scientifique}{\$2,000}{}
{}

\cventry{2020}{Methodological Development Scholarship}{Chaire de leadership en enseignement des sciences sociales numériques}{Université Laval}{}
{}

\cventry{2019}{Methodological Training Scholarship}{Centre pour l'étude de la citoyenneté démocratique}{Université McGill}{}
{}

\cventry{2019}{Methodological Training Scholarship}{Groupe de recherche en communication politique}{Université Laval}{}
{}

\cventry{2019}{Scholarship for Qualitative or Quantitative Methods Training}{Département de science politique}{Université Laval}{}
{}

\cventry{2019}{R Training Scholarship}{Département de science politique}{Université Laval}{}
{}

\cventry{2018}{Merit Scholarship}{Chaire de leadership en enseignement des sciences sociales numériques}{Université Laval}{}
{}


\vspace{1em}

%----------------------------------------------------------------------------------------
%	Specific Skills
%----------------------------------------------------------------------------------------

\section{\textbf{Specific Skills}}

\cvitem{Programming Languages}{R (advanced), HTML, CSS, LaTeX, Markdown}
\vspace{1em}

\cvitem{Tools and Software}{Git/Github, Notion (public template creator), Vim, Adobe Suite, Microsoft Office}
\vspace{1em}

\cvitem{Methodologies}{Automated text analysis (Bag of Words, Topic Modeling, LLM), Computational methods, Open science}
\vspace{1em}

\cvitem{Project Management}{Certified Scrum Master, Agile management, Notion Templates (\url{https://www.notion.com/@adri01})}
\vspace{1em}

\cvitem{Languages}{French (native) and English (fluent)}
\vspace{1em}

%----------------------------------------------------------------------------------------
%	References
%----------------------------------------------------------------------------------------

 \section{\textbf{References}}

 Available upon request.

%----------------------------------------------------------------------------------------

\end{document}
